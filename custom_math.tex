\makeatletter
\NewDocumentCommand{\swapcommands}{m m}{%
  \let\@swapcommands #1%
  \let #1 #2%
  \let #2\@swapcommands%
}
\makeatother

\makeatletter
\NewDocumentCommand{\wrapensuremath}{m}{%
  % Throw error, if "\N#1" is already defined.
  \expandafter\@ifdefinable\csname #1@ensuremath\endcsname{%
    % Save old meaning
    \expandafter
    \let\csname #1@ensuremath\expandafter\endcsname
    \csname #1\endcsname
    % Define new macro
    \expandafter\edef\csname #1\endcsname{%
      \noexpand\ensuremath{%
        \expandafter\noexpand\csname #1@ensuremath\endcsname
      }%
    }%
  }%
}
\makeatother

% Parse curly or round brackets
\makeatletter
\NewDocumentCommand{\@PCRB}{m m}{
  \IfNoValueTF{#1}{\IfNoValueTF{#2}{}{\quantity(#2)}}
  {#1 \IfNoValueTF{#2}{}{(#2)}}
}
% Parse math operator
\NewDocumentCommand{\parsemathoperator}{m m m}{
  \ensuremath{#1\@PCRB{#2}{#3}}
}
\makeatother

%=============================== Theorems ===============================
\newtheoremstyle{plainbreak}
{}%                                  % Space above, empty = `usual value'
{}%                                  % Space below
{\rmfamily}%                         % Body font
{}%                                  % Indent amount (empty = no indent, \parindent = para indent)
{\bfseries}%                         % Thm head font
{}%                                  % Punctuation after thm head
{\newline}%                          % Space after thm head: \newline = linebreak
{}%                                  % Thm head spec

\theoremstyle{plainbreak}
% Name in code, name in text, counter dependency
\IfLanguageName{russian}{
  \NewDocumentCommand{\TheoremName}{}{Теорема}
  \NewDocumentCommand{\PropositionName}{}{Утверждение}
  \NewDocumentCommand{\CorollaryName}{}{Следствие}
  \NewDocumentCommand{\LemmaName}{}{Лемма}
  \NewDocumentCommand{\ExampleName}{}{Пример}
  \NewDocumentCommand{\ExamplesName}{}{Примеры}
  \NewDocumentCommand{\ProblemName}{}{Упражнение}
  \NewDocumentCommand{\DefinitionName}{}{Определение}
  \NewDocumentCommand{\DenotationName}{}{Обозначение}
  \NewDocumentCommand{\RemarkName}{}{Замечание}
  \NewDocumentCommand{\SolutionName}{}{Решение}
}{}
\IfLanguageName{english}{
  \NewDocumentCommand{\TheoremName}{}{Theorem}
  \NewDocumentCommand{\PropositionName}{}{Proposition}
  \NewDocumentCommand{\CorollaryName}{}{Corollary}
  \NewDocumentCommand{\LemmaName}{}{Lemma}
  \NewDocumentCommand{\ExampleName}{}{Example}
  \NewDocumentCommand{\ExamplesName}{}{Examples}
  \NewDocumentCommand{\ProblemName}{}{Problem}
  \NewDocumentCommand{\DefinitionName}{}{Definition}
  \NewDocumentCommand{\DenotationName}{}{Denotation}
  \NewDocumentCommand{\RemarkName}{}{Remark}
  \NewDocumentCommand{\SolutionName}{}{Solution}
}{}
\newtheorem{theorem}{\TheoremName}[section]
% Counter here is placed in between, which means that it is not dependent but the same counter
\newtheorem{proposition}[theorem]{\PropositionName}
\newtheorem{corollary}{\CorollaryName}[theorem]
\newtheorem{lemma}[theorem]{\LemmaName}
\newtheorem*{example}{\ExampleName}
\newtheorem*{examples}{\ExamplesName}
\newtheorem{problem}{\ProblemName}[section]
\newtheorem{definition}{\DefinitionName}[section]
\newtheorem{denotation}[definition]{\DenotationName}

\theoremstyle{remark}
\newtheorem*{remark}{\RemarkName}
\newtheorem*{solution}{\SolutionName}

%=============================== Russian typography ===============================
\swapcommands{\epsilon}{\varepsilon}
\swapcommands{\phi}{\varphi}
\swapcommands{\kappa}{\varkappa}
\swapcommands{\emptyset}{\varnothing}
\swapcommands{\le}{\leqslant}
\swapcommands{\ge}{\geqslant}

%=============================== Ensure math for common symbols ===============================
% Uppercase versions are already defined
\wrapensuremath{alpha}
\wrapensuremath{beta}
\wrapensuremath{gamma}
\wrapensuremath{delta}
\wrapensuremath{epsilon}
\wrapensuremath{zeta}
\wrapensuremath{eta}
\wrapensuremath{theta}
\wrapensuremath{iota}
\wrapensuremath{kappa}
\wrapensuremath{lambda}
\wrapensuremath{mu}
\wrapensuremath{nu}
\wrapensuremath{xi}
\wrapensuremath{pi}
\wrapensuremath{rho}
\wrapensuremath{sigma}
\wrapensuremath{tau}
\wrapensuremath{upsilon}
\wrapensuremath{phi}
\wrapensuremath{chi}
\wrapensuremath{psi}
\wrapensuremath{omega}
\wrapensuremath{iff}
\wrapensuremath{rightarrow}
\wrapensuremath{leftarrow}
\wrapensuremath{implies}
\wrapensuremath{impliedby}

%=============================== Common math ===============================
% \mathcal and \mathbb or \mathscr fonts can be useful here

\NewDocumentCommand{\aqty}{m}{\ensuremath{\left<#1\right>}}
\NewDocumentCommand{\n}{m}{\ensuremath{\centernot{#1}}}
\NewDocumentCommand{\x}{}{\ensuremath{\times}}
\NewDocumentCommand{\ox}{}{\ensuremath{\otimes}}
\NewDocumentCommand{\transpose}{m}{#1^\top}
\NewDocumentCommand{\T}{}{^\top}
\NewDocumentCommand{\orthogonal}{}{^\top}
\NewDocumentCommand{\ort}{}{^\top}
\NewDocumentCommand{\herm}{}{^\dagger}
\NewDocumentCommand{\suchthat}{}{\colon}
\NewDocumentCommand{\func}{m m m}{\ensuremath{#1\colon #2\rightarrow #3}} % Function: X -> Y
\NewDocumentCommand{\isomorph}{}{\ensuremath{\cong}}
\NewDocumentCommand{\divby}{}{\mathrel{\rotatebox{90}{\ensuremath{\hskip-1pt.{}.{}.}}}}
\NewDocumentCommand{\ord}{g d()}{\parsemathoperator{\operatorname{ord}}{#1}{#2}}
\NewDocumentCommand{\sgn}{g d()}{\parsemathoperator{\operatorname{sgn}}{#1}{#2}}
\NewDocumentCommand{\Id }{g d()}{\parsemathoperator{\operatorname{Id}}{#1}{#2}}

%=============================== Sets ===============================
\NewDocumentCommand{\set}{m m}{\ensuremath{\qty{#1\colon #2}}}
\NewDocumentCommand{\powerset}{g d()}{\parsemathoperator{\mathcal{P}}{#1}{#2}}
\NewDocumentCommand{\permset}{g d()}{\parsemathoperator{\mathcal{S}}{#1}{#2}}
\NewDocumentCommand{\N}{}{\ensuremath{\mathbb{N}}}
\NewDocumentCommand{\Z}{}{\ensuremath{\mathbb{Z}}}
\NewDocumentCommand{\Q}{}{\ensuremath{\mathbb{Q}}}
\NewDocumentCommand{\R}{}{\ensuremath{\mathbb{R}}}
\RenewDocumentCommand{\C}{}{\ensuremath{\mathbb{C}}} % was U+030F "COMBINING DOUBLE GRAVE ACCENT".
\NewDocumentCommand{\union}{}{\ensuremath{\cup}}
\NewDocumentCommand{\intxn}{}{\ensuremath{\cap}}
\NewDocumentCommand{\Union}{}{\ensuremath{\bigcup}}
\NewDocumentCommand{\Intxn}{}{\ensuremath{\bigcap}}

%=============================== Mathematical logic ===============================
\NewDocumentCommand{\imply}{}{\ensuremath{\:\rightarrow\:}}
\NewDocumentCommand{\implied}{}{\ensuremath{\:\leftarrow\:}}
\NewDocumentCommand{\eqdef}{}{\ensuremath{\:\longleftrightarrow\:}}
\swapcommands{\equiv}{\eqdef}
\NewDocumentCommand{\iffdef}{}{\ensuremath{\aqty{\eqdef}}} % iff by definition
\NewDocumentCommand{\eqsym}{}{\ensuremath{\eqcirc}} % Equivalent symbol by symbol
\NewDocumentCommand{\A}{m g d()}{\parsemathoperator{\forall #1\:}{#2}{#3}}
\NewDocumentCommand{\E}{m g d()}{\parsemathoperator{\exists #1\:}{#2}{#3}}
\RenewDocumentCommand{\a}{}{\ensuremath{\land}}
\RenewDocumentCommand{\o}{}{\ensuremath{\lor}}   % was little emptyset


%=============================== Group theory ===============================
\RenewDocumentCommand{\ker}{g d()}{\parsemathoperator{\operatorname{ker}}{#1}{#2}} % improvement of previous ker
\NewDocumentCommand{\imag}{g d()}{\parsemathoperator{\operatorname{im}}{#1}{#2}}
\NewDocumentCommand{\subgr}{}{\ensuremath{\leqslant}}
\NewDocumentCommand{\rsubgr}{}{\ensuremath{\geqslant}}
\NewDocumentCommand{\normsubgr}{}{\ensuremath{\trianglelefteqslant}}
\NewDocumentCommand{\rnormsubgr}{}{\ensuremath{\trianglerighteqslant}}
\NewDocumentCommand{\grcenter}{g d()}{\parsemathoperator{\operatorname{Z}}{#1}{#2}}
\NewDocumentCommand{\centralizer}{m g d()}{\parsemathoperator{\operatorname{C}_{#1}}{#2}{#3}}
\NewDocumentCommand{\normalizer}{m g d()}{\parsemathoperator{\operatorname{N}_{#1}}{#2}{#3}}
\NewDocumentCommand{\factorgr}{m m}{\ensuremath{\nicefrac{#1}{#2}}}
\NewDocumentCommand{\rfactorgr}{m m}{%
  \reflectbox{%
    \nicefrac{\reflectbox{\ensuremath{#1}}}%
    {\reflectbox{\ensuremath{#2}}}}%
}

