\documentclass[a4paper,12pt]{article}

%=============================== Packages ===============================

\usepackage{iftex}
\ifXeTeX
    % Ensure you have CMU fonts installed!
    % Open Type, True Type, etc.
    \usepackage{fontspec}
    \defaultfontfeatures{Ligatures={TeX}, Renderer=Basic}
    \setmainfont[Ligatures={TeX,Historic}]{CMU Serif}
    \setsansfont{Times New Roman}
    \setmonofont{CMU Typewriter Text}
\else
    \RequirePDFTeX
    % Search in PDF document
    \usepackage{cmap}
    % Russian letters in formulas
    \usepackage{mathtext}
    % Encodings
    \usepackage[T2A]{fontenc}
    \usepackage[utf8]{inputenc}
\fi
\usepackage[english,russian]{babel}

% NewDocumentCommand and conditional expressions
\usepackage{xparse}

\usepackage[usenames,dvipsnames,svgnames,table,rgb]{xcolor}

\usepackage{enumitem}

\usepackage{multicol}

% Margins
\usepackage{geometry}

% Tables
\usepackage{array,booktabs}
\usepackage{tabularx,tabulary}
\usepackage{longtable}
\usepackage{multirow}

% Pictures
\usepackage{graphicx}
\usepackage{wrapfig}
\usepackage{epstopdf}
\usepackage{adjustbox}

\usepackage{hyperref}

% Correct quotation
\usepackage{fvextra}
\usepackage{csquotes}

% First line indent
\usepackage{indentfirst}
\usepackage{parskip}

% Symbols like copyright, degree
\usepackage{textcomp}

% Bold, italics, etc.
\usepackage{soul}

% Number of pages in document
\usepackage{lastpage}

% Header and footer
\usepackage{fancyhdr}

% Linespacing
\usepackage{setspace}

\usepackage{amsmath}

% Extended math symbol collection
\usepackage{amssymb}

\usepackage{amsfonts}

\usepackage{amsthm}

% Fix bugs in ams
\usepackage{mathtools}

% Math font with \mathscr command
\usepackage{mathrsfs}

% Additional math symbols
\usepackage{stmaryrd}

% Code hightlighting
\usepackage[outputdir=build]{minted}

% A comprehensive SI units package
\usepackage{siunitx}

% \nicefrac \unitfrac and \unit command
\usepackage{units}

% \qty, \norm, \abs, \div, \grad, \curl, \laplacian, etc.
\usepackage{physics}

% Graphics
\usepackage{tikz}
\usetikzlibrary{matrix}
\usepackage{pgfplots}
\pgfplotsset{compat=1.17}
\usepackage{pgfplotstable}


%=============================== Customizations ===============================

% Set serif default font
%\renewcommand{\familydefault}{\sfdefault}

% Set linespacing
%\singlespacing
%\onehalfspacing
%\doublespacing

% Set up margins
\geometry{top=20mm}
\geometry{bottom=20mm}
\geometry{left=10mm}
\geometry{right=10mm}

% Set up headers
\pagestyle{fancy}
\renewcommand{\headrulewidth}{0mm}
\lfoot{}
\rfoot{}
\rhead{}
\chead{}
\lhead{}
%\cfoot{Нижний в центре}

\graphicspath{{images/}{build/images/}}

% Picture box margins
\setlength\fboxsep{3pt}
\setlength\fboxrule{1pt}

% No first line indent
\setlength{\parindent}{0ex}
\frenchspacing

\hypersetup{
    unicode=true,
    pdftitle={Домашнее задание \#1},
    pdfauthor={Никита Руденко},
    % False means refs in boxes, true means color refs
    colorlinks=true,
    linkcolor=red,
    citecolor=green,
    filecolor=magenta,
    urlcolor=blue
}

% Only show numbers if formula is referenced
%\mathtoolsset{showonlyrefs=true}

% Display formula number on the left
%\usepackage{leqno}


%=============================== Support commands ===============================
\makeatletter
\NewDocumentCommand{\swapcommands}{m m}{%
    \let\@swapcommands #1%
    \let #1 #2%
    \let #2\@swapcommands%
}
\NewDocumentCommand{\wrapensuremath}{m}{%
    % Throw error, if "\N#1" is already defined.
    \expandafter\@ifdefinable\csname #1@ensuremath\endcsname{%
        % Save old meaning
        \expandafter
        \let\csname #1@ensuremath\expandafter\endcsname
        \csname #1\endcsname
        % Define new macro
        \expandafter\edef\csname #1\endcsname{%
            \noexpand\ensuremath{%
                \expandafter\noexpand\csname #1@ensuremath\endcsname
            }%
        }%
    }%
}

% Parse curly or round brackets
\NewDocumentCommand{\@PCRB}{m m}{
    \IfNoValueTF{#1}{\IfNoValueTF{#2}{}{\quantity(#2)}}
    {#1 \IfNoValueTF{#2}{}{(#2)}}
}
% Parse math operator
\NewDocumentCommand{\parsemathoperator}{m m m}{
    \ensuremath{#1\@PCRB{#2}{#3}}
}




%=============================== Russian typography ===============================
\NewDocumentCommand{\varle}{}{\leqslant}
\NewDocumentCommand{\varge}{}{\geqslant}
\swapcommands{\epsilon}{\varepsilon}
\swapcommands{\phi}{\varphi}
\swapcommands{\kappa}{\varkappa}
\swapcommands{\emptyset}{\varnothing}
\swapcommands{\le}{\varle}
\swapcommands{\ge}{\varge}

%=============================== Ensure math for common symbols ===============================
\wrapensuremath{alpha}
\wrapensuremath{beta}
\wrapensuremath{gamma}   \wrapensuremath{Gamma}
\wrapensuremath{delta}   \wrapensuremath{Delta}
\wrapensuremath{epsilon}
\wrapensuremath{zeta}
\wrapensuremath{eta}
\wrapensuremath{theta}   \wrapensuremath{Theta}
\wrapensuremath{iota}
\wrapensuremath{kappa}
\wrapensuremath{lambda}  \wrapensuremath{Lambda}
\wrapensuremath{mu}
\wrapensuremath{nu}
\wrapensuremath{xi}      \wrapensuremath{Xi}
\wrapensuremath{pi}      \wrapensuremath{Pi}
\wrapensuremath{rho}
\wrapensuremath{sigma}   \wrapensuremath{Sigma}
\wrapensuremath{tau}
\wrapensuremath{upsilon} \wrapensuremath{Upsilon}
\wrapensuremath{phi}     \wrapensuremath{Phi}
\wrapensuremath{chi}
\wrapensuremath{psi}     \wrapensuremath{Psi}
\wrapensuremath{omega}   \wrapensuremath{Omega}
\wrapensuremath{iff}
\wrapensuremath{rightarrow}
\wrapensuremath{leftarrow}
\wrapensuremath{implies}
\wrapensuremath{impliedby}

%=============================== Common math ===============================
\NewDocumentCommand{\aqty}{m}{\ensuremath{\left<#1\right>}}
\NewDocumentCommand{\n}{m}{\ensuremath{\centernot{#1}}}
\NewDocumentCommand{\x}{}{\ensuremath{\times}}
\NewDocumentCommand{\ox}{}{\ensuremath{\otimes}}
\NewDocumentCommand{\transpose}{m}{#1^\top}
\NewDocumentCommand{\T}{}{^\top}
\NewDocumentCommand{\orthogonal}{}{^\top}
\NewDocumentCommand{\ort}{}{^\top}
\NewDocumentCommand{\herm}{}{^\dagger}
\NewDocumentCommand{\suchthat}{}{\colon}
\NewDocumentCommand{\func}{m m m}{\ensuremath{#1\colon #2\rightarrow #3}} % Function: X -> Y
\NewDocumentCommand{\isomorph}{}{\ensuremath{\cong}}
\NewDocumentCommand{\divby}{}{\mathrel{\rotatebox{90}{\ensuremath{\hskip-1pt.{}.{}.}}}}
\NewDocumentCommand{\ord}{g d()}{\parsemathoperator{\operatorname{ord}}{#1}{#2}}
\NewDocumentCommand{\sgn}{g d()}{\parsemathoperator{\operatorname{sgn}}{#1}{#2}}
\NewDocumentCommand{\Id }{g d()}{\parsemathoperator{\operatorname{Id}}{#1}{#2}}

%=============================== Sets ===============================
\NewDocumentCommand{\set}{m m}{\ensuremath{\qty{#1\colon #2}}}
\NewDocumentCommand{\powerset}{g d()}{\parsemathoperator{\mathcal{P}}{#1}{#2}}
\NewDocumentCommand{\permset}{g d()}{\parsemathoperator{\mathcal{S}}{#1}{#2}}
\NewDocumentCommand{\N}{}{\ensuremath{\mathbb{N}}}
\NewDocumentCommand{\Z}{}{\ensuremath{\mathbb{Z}}}
\NewDocumentCommand{\Q}{}{\ensuremath{\mathbb{Q}}}
\NewDocumentCommand{\R}{}{\ensuremath{\mathbb{R}}}
\RenewDocumentCommand{\C}{}{\ensuremath{\mathbb{C}}} % was U+030F "COMBINING DOUBLE GRAVE ACCENT".
\NewDocumentCommand{\union}{}{\ensuremath{\cup}}
\NewDocumentCommand{\intxn}{}{\ensuremath{\cap}}
\NewDocumentCommand{\Union}{}{\ensuremath{\bigcup}}
\NewDocumentCommand{\Intxn}{}{\ensuremath{\bigcap}}

%=============================== Mathematical logic ===============================
\NewDocumentCommand{\imply}{}{\ensuremath{\:\rightarrow\:}}
\NewDocumentCommand{\implied}{}{\ensuremath{\:\leftarrow\:}}
\NewDocumentCommand{\eqdef}{}{\ensuremath{\:\longleftrightarrow\:}}
\swapcommands{\equiv}{\eqdef}
\NewDocumentCommand{\iffdef}{}{\ensuremath{\aqty{\eqdef}}} % iff by definition
\NewDocumentCommand{\eqsym}{}{\ensuremath{\eqcirc}} % Equivalent symbol by symbol
\NewDocumentCommand{\A}{m g d()}{\parsemathoperator{\forall #1\:}{#2}{#3}}
\NewDocumentCommand{\E}{m g d()}{\parsemathoperator{\exists #1\:}{#2}{#3}}
\RenewDocumentCommand{\a}{}{\ensuremath{\land}}
\RenewDocumentCommand{\o}{}{\ensuremath{\lor}}   % was little emptyset


%=============================== Group theory ===============================
\RenewDocumentCommand{\ker}{g d()}{\parsemathoperator{\operatorname{ker}}{#1}{#2}} % improvement of previous ker
\NewDocumentCommand{\imag}{g d()}{\parsemathoperator{\operatorname{im}}{#1}{#2}}
\NewDocumentCommand{\subgr}{}{\ensuremath{\leqslant}}
\NewDocumentCommand{\rsubgr}{}{\ensuremath{\geqslant}}
\NewDocumentCommand{\normsubgr}{}{\ensuremath{\trianglelefteqslant}}
\NewDocumentCommand{\rnormsubgr}{}{\ensuremath{\trianglerighteqslant}}
\NewDocumentCommand{\grcenter}{g d()}{\parsemathoperator{\operatorname{Z}}{#1}{#2}}
\NewDocumentCommand{\centralizer}{m g d()}{\parsemathoperator{\operatorname{C}_{#1}}{#2}{#3}}
\NewDocumentCommand{\normalizer}{m g d()}{\parsemathoperator{\operatorname{N}_{#1}}{#2}{#3}}
\NewDocumentCommand{\factorgr}{m m}{\ensuremath{\nicefrac{#1}{#2}}}
\NewDocumentCommand{\rfactorgr}{m m}{%
    \reflectbox{%
        \nicefrac{\reflectbox{\ensuremath{#1}}}%
        {\reflectbox{\ensuremath{#2}}}}%
}




%=============================== Theorems ===============================
\newtheoremstyle{plainbreak}
{}%                                  % Space above, empty = `usual value'
{}%                                  % Space below
{\rmfamily}%                         % Body font
{}%                                  % Indent amount (empty = no indent, \parindent = para indent)
{\bfseries}%                         % Thm head font
{}%                                  % Punctuation after thm head
{\newline}%                          % Space after thm head: \newline = linebreak
{}%                                  % Thm head spec

\theoremstyle{plainbreak}
% Name in code, name in text, counter dependency
\newtheorem{theorem}{Теорема}[section]
% Counter here is placed in between, which means that it is not dependent but the same counter
\newtheorem{proposition}[theorem]{Утверждение}
\newtheorem{corollary}{Следствие}[theorem]
\newtheorem{lemma}[theorem]{Лемма}
\newtheorem*{example}{Пример}
\newtheorem*{examples}{Примеры}
\newtheorem{problem}{Задача}[section]
\newtheorem{definition}{Определение}[section]
\newtheorem{denotation}[definition]{Обозначение}

\theoremstyle{remark}
\newtheorem*{remark}{Замечание}
\newtheorem*{solution}{Решение}


\newcommand*\@proofenvname{proof}
\newcommand*{\theoremlistshack}{%
    \leavevmode
    \ifx\@currenvir\@proofenvname
    \else
        \vspace{-\baselineskip}
    \fi
    \par%
    \everypar{\setbox\z@\lastbox\everypar{}}%
}

\makeatother

\makeatletter
\NewDocumentCommand{\swapcommands}{m m}{%
  \let\@swapcommands #1%
  \let #1 #2%
  \let #2\@swapcommands%
}
\makeatother

\makeatletter
\NewDocumentCommand{\wrapensuremath}{m}{%
  % Throw error, if "\N#1" is already defined.
  \expandafter\@ifdefinable\csname #1@ensuremath\endcsname{%
    % Save old meaning
    \expandafter
    \let\csname #1@ensuremath\expandafter\endcsname
    \csname #1\endcsname
    % Define new macro
    \expandafter\edef\csname #1\endcsname{%
      \noexpand\ensuremath{%
        \expandafter\noexpand\csname #1@ensuremath\endcsname
      }%
    }%
  }%
}
\makeatother

% Parse curly or round brackets
\makeatletter
\NewDocumentCommand{\@PCRB}{m m}{
  \IfNoValueTF{#1}{\IfNoValueTF{#2}{}{\quantity(#2)}}
  {#1 \IfNoValueTF{#2}{}{(#2)}}
}
% Parse math operator
\NewDocumentCommand{\parsemathoperator}{m m m}{
  \ensuremath{#1\@PCRB{#2}{#3}}
}
\makeatother

%=============================== Theorems ===============================
\newtheoremstyle{plainbreak}
{}%                                  % Space above, empty = `usual value'
{}%                                  % Space below
{\rmfamily}%                         % Body font
{}%                                  % Indent amount (empty = no indent, \parindent = para indent)
{\bfseries}%                         % Thm head font
{}%                                  % Punctuation after thm head
{\newline}%                          % Space after thm head: \newline = linebreak
{}%                                  % Thm head spec

\theoremstyle{plainbreak}
% Name in code, name in text, counter dependency
\IfLanguageName{russian}{
  \NewDocumentCommand{\TheoremName}{}{Теорема}
  \NewDocumentCommand{\PropositionName}{}{Утверждение}
  \NewDocumentCommand{\CorollaryName}{}{Следствие}
  \NewDocumentCommand{\LemmaName}{}{Лемма}
  \NewDocumentCommand{\ExampleName}{}{Пример}
  \NewDocumentCommand{\ExamplesName}{}{Примеры}
  \NewDocumentCommand{\ProblemName}{}{Упражнение}
  \NewDocumentCommand{\DefinitionName}{}{Определение}
  \NewDocumentCommand{\DenotationName}{}{Обозначение}
  \NewDocumentCommand{\RemarkName}{}{Замечание}
  \NewDocumentCommand{\SolutionName}{}{Решение}
}{}
\IfLanguageName{english}{
  \NewDocumentCommand{\TheoremName}{}{Theorem}
  \NewDocumentCommand{\PropositionName}{}{Proposition}
  \NewDocumentCommand{\CorollaryName}{}{Corollary}
  \NewDocumentCommand{\LemmaName}{}{Lemma}
  \NewDocumentCommand{\ExampleName}{}{Example}
  \NewDocumentCommand{\ExamplesName}{}{Examples}
  \NewDocumentCommand{\ProblemName}{}{Problem}
  \NewDocumentCommand{\DefinitionName}{}{Definition}
  \NewDocumentCommand{\DenotationName}{}{Denotation}
  \NewDocumentCommand{\RemarkName}{}{Remark}
  \NewDocumentCommand{\SolutionName}{}{Solution}
}{}
\newtheorem{theorem}{\TheoremName}[section]
% Counter here is placed in between, which means that it is not dependent but the same counter
\newtheorem{proposition}[theorem]{\PropositionName}
\newtheorem{corollary}{\CorollaryName}[theorem]
\newtheorem{lemma}[theorem]{\LemmaName}
\newtheorem*{example}{\ExampleName}
\newtheorem*{examples}{\ExamplesName}
\newtheorem{problem}{\ProblemName}[section]
\newtheorem{definition}{\DefinitionName}[section]
\newtheorem{denotation}[definition]{\DenotationName}

\theoremstyle{remark}
\newtheorem*{remark}{\RemarkName}
\newtheorem*{solution}{\SolutionName}

%=============================== Russian typography ===============================
\swapcommands{\epsilon}{\varepsilon}
\swapcommands{\phi}{\varphi}
\swapcommands{\kappa}{\varkappa}
\swapcommands{\emptyset}{\varnothing}
\swapcommands{\le}{\leqslant}
\swapcommands{\ge}{\geqslant}

%=============================== Ensure math for common symbols ===============================
% Uppercase versions are already defined
\wrapensuremath{alpha}
\wrapensuremath{beta}
\wrapensuremath{gamma}
\wrapensuremath{delta}
\wrapensuremath{epsilon}
\wrapensuremath{zeta}
\wrapensuremath{eta}
\wrapensuremath{theta}
\wrapensuremath{iota}
\wrapensuremath{kappa}
\wrapensuremath{lambda}
\wrapensuremath{mu}
\wrapensuremath{nu}
\wrapensuremath{xi}
\wrapensuremath{pi}
\wrapensuremath{rho}
\wrapensuremath{sigma}
\wrapensuremath{tau}
\wrapensuremath{upsilon}
\wrapensuremath{phi}
\wrapensuremath{chi}
\wrapensuremath{psi}
\wrapensuremath{omega}
\wrapensuremath{iff}
\wrapensuremath{rightarrow}
\wrapensuremath{leftarrow}
\wrapensuremath{implies}
\wrapensuremath{impliedby}

%=============================== Common math ===============================
% \mathcal and \mathbb or \mathscr fonts can be useful here

\NewDocumentCommand{\aqty}{m}{\ensuremath{\left<#1\right>}}
\NewDocumentCommand{\n}{m}{\ensuremath{\centernot{#1}}}
\NewDocumentCommand{\x}{}{\ensuremath{\times}}
\NewDocumentCommand{\ox}{}{\ensuremath{\otimes}}
\NewDocumentCommand{\transpose}{m}{#1^\top}
\NewDocumentCommand{\T}{}{^\top}
\NewDocumentCommand{\orthogonal}{}{^\top}
\NewDocumentCommand{\ort}{}{^\top}
\NewDocumentCommand{\herm}{}{^\dagger}
\NewDocumentCommand{\suchthat}{}{\colon}
\NewDocumentCommand{\func}{m m m}{\ensuremath{#1\colon #2\rightarrow #3}} % Function: X -> Y
\NewDocumentCommand{\isomorph}{}{\ensuremath{\cong}}
\NewDocumentCommand{\divby}{}{\mathrel{\rotatebox{90}{\ensuremath{\hskip-1pt.{}.{}.}}}}
\NewDocumentCommand{\ord}{g d()}{\parsemathoperator{\operatorname{ord}}{#1}{#2}}
\NewDocumentCommand{\sgn}{g d()}{\parsemathoperator{\operatorname{sgn}}{#1}{#2}}
\NewDocumentCommand{\Id }{g d()}{\parsemathoperator{\operatorname{Id}}{#1}{#2}}

%=============================== Sets ===============================
\NewDocumentCommand{\set}{m m}{\ensuremath{\qty{#1\colon #2}}}
\NewDocumentCommand{\powerset}{g d()}{\parsemathoperator{\mathcal{P}}{#1}{#2}}
\NewDocumentCommand{\permset}{g d()}{\parsemathoperator{\mathcal{S}}{#1}{#2}}
\NewDocumentCommand{\N}{}{\ensuremath{\mathbb{N}}}
\NewDocumentCommand{\Z}{}{\ensuremath{\mathbb{Z}}}
\NewDocumentCommand{\Q}{}{\ensuremath{\mathbb{Q}}}
\NewDocumentCommand{\R}{}{\ensuremath{\mathbb{R}}}
\RenewDocumentCommand{\C}{}{\ensuremath{\mathbb{C}}} % was U+030F "COMBINING DOUBLE GRAVE ACCENT".
\NewDocumentCommand{\union}{}{\ensuremath{\cup}}
\NewDocumentCommand{\intxn}{}{\ensuremath{\cap}}
\NewDocumentCommand{\Union}{}{\ensuremath{\bigcup}}
\NewDocumentCommand{\Intxn}{}{\ensuremath{\bigcap}}

%=============================== Mathematical logic ===============================
\NewDocumentCommand{\imply}{}{\ensuremath{\:\rightarrow\:}}
\NewDocumentCommand{\implied}{}{\ensuremath{\:\leftarrow\:}}
\NewDocumentCommand{\eqdef}{}{\ensuremath{\:\longleftrightarrow\:}}
\swapcommands{\equiv}{\eqdef}
\NewDocumentCommand{\iffdef}{}{\ensuremath{\aqty{\eqdef}}} % iff by definition
\NewDocumentCommand{\eqsym}{}{\ensuremath{\eqcirc}} % Equivalent symbol by symbol
\NewDocumentCommand{\A}{m g d()}{\parsemathoperator{\forall #1\:}{#2}{#3}}
\NewDocumentCommand{\E}{m g d()}{\parsemathoperator{\exists #1\:}{#2}{#3}}
\RenewDocumentCommand{\a}{}{\ensuremath{\land}}
\RenewDocumentCommand{\o}{}{\ensuremath{\lor}}   % was little emptyset


%=============================== Group theory ===============================
\RenewDocumentCommand{\ker}{g d()}{\parsemathoperator{\operatorname{ker}}{#1}{#2}} % improvement of previous ker
\NewDocumentCommand{\imag}{g d()}{\parsemathoperator{\operatorname{im}}{#1}{#2}}
\NewDocumentCommand{\subgr}{}{\ensuremath{\leqslant}}
\NewDocumentCommand{\rsubgr}{}{\ensuremath{\geqslant}}
\NewDocumentCommand{\normsubgr}{}{\ensuremath{\trianglelefteqslant}}
\NewDocumentCommand{\rnormsubgr}{}{\ensuremath{\trianglerighteqslant}}
\NewDocumentCommand{\grcenter}{g d()}{\parsemathoperator{\operatorname{Z}}{#1}{#2}}
\NewDocumentCommand{\centralizer}{m g d()}{\parsemathoperator{\operatorname{C}_{#1}}{#2}{#3}}
\NewDocumentCommand{\normalizer}{m g d()}{\parsemathoperator{\operatorname{N}_{#1}}{#2}{#3}}
\NewDocumentCommand{\factorgr}{m m}{\ensuremath{\nicefrac{#1}{#2}}}
\NewDocumentCommand{\rfactorgr}{m m}{%
  \reflectbox{%
    \nicefrac{\reflectbox{\ensuremath{#1}}}%
    {\reflectbox{\ensuremath{#2}}}}%
}


\usepackage{lipsum}

\global\mdfdefinestyle{example}{%
  leftline=false,%
  rightline=true,%
  topline=false,%
  bottomline=false,%
  innerleftmargin=0ex,%
  linecolor=red,%
  linewidth=2pt
}

\author{Nikita Rudenko}
\title{\LaTeX{} notes}
\date{\the\year{}}

\begin{document}
\maketitle

\renewcommand{\abstractname}{Abstract}
\begin{abstract}
  \LaTeX{} notes heavily inspired by \autocite{DanilFedorovykh-2016}.
\end{abstract}

\section{Fonts and alignment}
\subsection{Alignment}
\thispagestyle{empty}
Center, left and right alignment with standard \mintinline{tex}{center}, \mintinline{tex}{flushright} and \mintinline{tex}{flushleft} environments.
\begin{minted}{tex}
  \begin{center}
    Some text
  \end{center}
\end{minted}
\begin{mdframed}[style=example]
  \begin{center}
    Some text
  \end{center}
\end{mdframed}

Add \mintinline{tex}{\noindent} command into these environments for disabling indent for the first line locally (if it is on).
Insert vertical space with \mintinline{tex}{\vspace{3ex}} command.
And create several columns with \mintinline{tex}{multicols} environment from \mintinline{tex}{multicol} package.
\begin{minted}{tex}
  \begin{multicols}{<number or columns>}
    <Some text>
  \end{multicols}
\end{minted}

\begin{mdframed}[style=example]
  \begin{multicols}{2}
    First column

    Second column
  \end{multicols}
\end{mdframed}

Alignment, vertical spaces, local page styles are useful for creating first page.
Remove headers and footers only on this page with \mintinline{tex}{\thispagestyle{empty}}.
See also \mintinline{tex}{fancyhdr} package.
Page break can be done with \mintinline{tex}{\newpage}
Another command is \mintinline{tex}{\vfill}, which inserts vertical space to the end of page, except for the text following this command, \ie{} text is growing "upwards".
\begin{minted}{tex}
  \vfill
  \begin{center}
      MIPT
  \end{center}
\end{minted}
\vfill
\begin{mdframed}[style=example]
  \begin{center}
    MIPT
  \end{center}
\end{mdframed}


\subsection{Font size}
There three different ways of changing relative font size.

\begin{enumerate}
  \item Directive, which affects the whole document after it: \mintinline{tex}{\small}
  \item Inline, which affects only the text in braces: \mintinline{tex}{{\small text}}
  \item And environment:
        \begin{minted}[framesep=-10ex]{tex}
          \begin{small}
            text
          \end{small}
        \end{minted}
\end{enumerate}

List of relative font sizes:
\begin{itemize}
  \item {\tiny tiny}
  \item {\scriptsize scriptsize}
  \item {\footnotesize footnotesize}
  \item {\small small}
  \item {\normalsize normalsize}
  \item {\large large}
  \item {\Large Large}
  \item {\LARGE LARGE}
  \item {\huge huge}
  \item {\Huge Huge}
\end{itemize}

\subsection{Emphasis}
\begin{itemize}
  \item \textbf{Bold}: \mintinline{tex}{\textbf{bold}}
  \item \textit{Italics}: \mintinline{tex}{\textit{italics}}
  \item \emph{Emphasized}: \mintinline{tex}{\emph{emphasized}}
\end{itemize}

\section{Typography}
\subsection{Dashes}
\begin{itemize}
  \item N dash (--) \autocite{thesauruscom-en-dash-2022}: for ranges, such as years, pages, sports scores, etc.
        \begin{minted}[framesep=-10ex]{tex}
          Homework problems are on pages 10--15.
          The home team won 2–-0.
        \end{minted}
        \begin{mdframed}[style=example]
          Homework problems are on pages 10--15.
          The home team won 2–-0.
        \end{mdframed}
  \item M dash (---) \autocite{thesauruscom-em-dash-2022}: common dash for missing sentence members. Can be in form of \verb|---| or \verb|"---|.
        The latter is used in russian typesetting system.
  \item Direct speech: \verb|"--*|
\end{itemize}

\subsection{Quotation marks}
See articles on \emph{Overleaf} \autocite{Overleaf-quotes-2018} and \emph{Wikipedia} \autocite{Wikipedia-quotes-2006}.
\begin{itemize}
  \item "text"
  \item <<text>>
  \item ,,text``
\end{itemize}

\subsection{Spaces}
From \citetitle{JohannesWienke-2018} \autocite{JohannesWienke-2018}:
\LaTeX{} distinguishes between inter-word and inter-sentence spaces.
Generally, a space is treated as an inter-word space as long as the character before the space is not a period.
Thus, if you use an abbreviation with a period at the end, you have to instruct \LaTeX{} that this period does not end the current sentence by escaping the following space:
\begin{minted}{tex}
  This is a sentence w.\ a forced inter-word space.
\end{minted}

If you don’t do this, you randomly get larger spaces after abbreviations that interrupt the visual flow.
One exception to this rule is if the period is preceded by a capital letter.
In this case, \LaTeX{} assumes an abbreviation and continues to use an inter-word space.
You can reverse this behavior by placing an \textbackslash @ before the period.
\begin{minted}{tex}
  This is done by XYZ\@. This is a new sentence.
\end{minted}

\LaTeX{} also has short space (\mintinline{tex}{\:}) and non-breakable space (\mintinline{tex}{~}).

\section{Lists}
Lists can be created with standard \mintinline{tex}{itemize} environment or \mintinline{tex}{enumerate} environment from \mintinline{tex}{enumitem} package.
\begin{minted}{tex}
  \begin{itemize}
    \item First point
    \item Second point
    \item[*] Third point with custom bullet
  \end{itemize}
\end{minted}

\begin{mdframed}[style=example]
  \begin{itemize}
    \item First point
    \item Second point
    \item[*] Third point with custom bullet
  \end{itemize}
\end{mdframed}

\section{Images}
Images can be included with \mintinline{tex}{\includegraphics} (requires \mintinline{tex}{graphicx} package), \mintinline{tex}{\adjincludegraphics} (requires \mintinline{tex}{graphicx} and \mintinline{tex}{adjustbox} packages) or \mintinline{tex}{\adjustimage} (also \mintinline{tex}{graphicx} and \mintinline{tex}{adjustbox} packages) commands.
All three can scale images and the latter two can crop them.
It is possible, but usually is not recommended to set up automatic conversion between different image sources and \mintinline{tex}{pdf}.
Instead, it is better to use build system, such as \mintinline{tex}{make}.
Conversion between \mintinline{tex}{eps} and \mintinline{tex}{pdf} is enabled though.

\begin{minted}{tex}
  \includegraphics[width=\textwidth/2]{drawio_example.pdf}
  \adjincludegraphics[width=\textwidth/2, trim={{0.02\width} {.95\height} {.91\width} {0}},clip]{mathcha_example.pdf}
  \adjustimage{max width=\textwidth/2, trim={{.6\width} {.45\height} {.25\width} {.43\height}},clip}{geogebra_example.pdf}
\end{minted}

\begin{figure}[!h]
  \centering
  \hfill
  \begin{subfigure}[t]{0.3\textwidth}
    \centering
    \includegraphics[width=\textwidth/2]{drawio_example.pdf}
    \caption{draw.io example}
  \end{subfigure}
  \begin{subfigure}[t]{0.3\textwidth}
    \centering
    \adjincludegraphics[width=\textwidth/2, trim={{0.02\width} {.95\height} {.91\width} {0}},clip]{mathcha_example.pdf}
    \caption{Mathcha example}
  \end{subfigure}
  \begin{subfigure}[t]{0.3\textwidth}
    \centering
    \adjustimage{max width=\textwidth/2, trim={{.6\width} {.45\height} {.25\width} {.43\height}},clip}{geogebra_example.pdf}
    \caption{GeoGebra example}
  \end{subfigure}
  \caption{Figure examples}
\end{figure}

There is a bunch of useful tools for math drawings and diagrams, such as \href{https://www.geogebra.org/}{GeoGebra}, \href{https://asymptote.sourceforge.io/}{Asymptote}, \href{https://www.mathcha.io/}{Mathcha}, \href{https://app.diagrams.net/}{draw.io}, \href{https://tikzcd.yichuanshen.de/}{tikzcd-editor}, \href{https://inkscape.org/}{Inkscape}, etc.
Some of them support exporting to tikz or \LaTeX{} format.

GeoGebra may use outdated syntax for circles drawing \autocite{Aalberto-2020}.
This can be easily fixed in vim using the following command:
\begin{minted}{vim}
  :%s/circle (\(\%(\d\|\.\)\+\w\+\))/circle [radius=\1]/g
\end{minted}

\section{Tables}
Tables can be created with standard \mintinline{tex}{tabular} environment or with \mintinline{tex}{tabularx}, \mintinline{tex}{tabulary} and \mintinline{tex}{longtable} environments from packages of the same name.
\mintinline{tex}{tabularx} can justify columns by width, \mintinline{tex}{tabulary} by height and \mintinline{tex}{longtable} specializes on creating well\dots\ long tables.
Nice formatting for tables can be done with \mintinline{tex}{booktabs} package.

Tabular objects can be put into some environment, usually \mintinline{tex}{table}, which finds a suitable place for them.
It is also possible to wrap text around tables and images by placing them into \mintinline{tex}{wraptable} and \mintinline{tex}{wrapfigure} environments from \mintinline{tex}{wrapfig} package.

\begin{minted}{tex}
  \begin{table}
    \centering
    \label{tab:table_example}
    \begin{tabular}{lll}
      \toprule
            & Column 1 & Column 2 \\
      \midrule
      Row 1 & 1-1      & 1-2      \\
      Row 2 & 2-1      & 2-2      \\
      \bottomrule
    \end{tabular}
    \caption{Table example}
  \end{table}
\end{minted}

\begin{table}[h!]
  \centering
  \label{tab:table_example}
  \begin{tabular}{lll}
    \toprule
          & Column 1 & Column 2 \\
    \midrule
    Row 1 & 1-1      & 1-2      \\
    Row 2 & 2-1      & 2-2      \\
    \bottomrule
  \end{tabular}
  \caption{Table example}
\end{table}

There is a nice tutorial on \emph{overleaf} \autocite{Overleaf-tables-2018} and several useful tools for generating tables, like \href{https://www.tablesgenerator.com/latex_tables}{this} and \href{https://www.latex-tables.com/}{this}.

\section{Math}
\subsection{Math modes}
There are different ways of entering math mode:
\begin{itemize}
  \item Inline math mode with \mintinline{tex}{$$}, for example \mintinline{tex}{$2+2=4$}: $2+2=4$. %stopzone
        Inline mode tries to adjust formula size to the size of text.
        It can be overridden with \mintinline{tex}{\displaystyle} command:
        \begin{minted}[framesep=-10ex]{tex}
          For example, inline $\frac{x}{y}$ and inline $\displaystyle \frac{x}{y}$.
        \end{minted}
        \begin{mdframed}[style=example]
          For example, inline $\frac{x}{y}$ and inline $\displaystyle \frac{x}{y}$.
        \end{mdframed}
  \item Outline with \mintinline{tex}{\[\]}, for example \mintinline{tex}{\[2+2=4\]}
        \begin{mdframed}[style=example]
          \[2+2=4\]
        \end{mdframed}
  \item Environment with \mintinline{tex}{equation}.
        \begin{minted}[framesep=-10ex]{tex}
          \begin{equation}
            2+2=4
          \end{equation}
        \end{minted}
        \begin{mdframed}[style=example]
          \begin{equation}
            2+2=4
          \end{equation}
        \end{mdframed}
\end{itemize}

\mintinline{tex}{equation} and some other similar environments automatically add tag for referencing, as in the example above.
This usually can be turned off with adding asterisk to environment name:
\begin{minted}{tex}
  \begin{equation*}
    2+2=4
  \end{equation*}
\end{minted}
\begin{mdframed}[style=example]
  \begin{equation*}
    2+2=4
  \end{equation*}
\end{mdframed}

To actually be able to reference equation it has to have label on it.
Also, it is possible to specify custom tag:
\begin{minted}{tex}
  \begin{equation} \tag{TAG} \label{eq:2}
    2+2=4
  \end{equation}
\end{minted}
\begin{mdframed}[style=example]
  \begin{equation} \tag{TAG} \label{eq:2}
    2+2=4
  \end{equation}
\end{mdframed}

\begin{minted}{tex}
  See equation \eqref{eq:2} on page \pageref{eq:2}.
\end{minted}
\begin{mdframed}[style=example]
  See equation \eqref{eq:2} on page \pageref{eq:2}.
\end{mdframed}

\subsection{Equation grouping}
\mintinline{tex}{align} environment triggers math mode automatically.
On the other hand \mintinline{tex}{alined} environment does not, but it can be a part of more complex structures.
Both of these environments align elements in one column using \mintinline{tex}{&} hint, \ie{} odd \mintinline{tex}{&}'s are used for alignment and even \mintinline{tex}{&}'s are used for separating columns.
\begin{minted}{tex}
  \begin{align}
    1\times 1 & = 1 & 4\times 4     & = 16    \\
    2\times 2 & = 4 & 5\times 5     & = 25    \\
    3\times 3 & = 9 & 60\times 60   & = 3600
  \end{align}
\end{minted}
\begin{mdframed}[style=example]
  \begin{align}
    1\times 1 & = 1 & 4\times 4   & = 16   \\
    2\times 2 & = 4 & 5\times 5   & = 25   \\
    3\times 3 & = 9 & 60\times 60 & = 3600
  \end{align}
\end{mdframed}

\subsection{Fractions}
\mintinline{tex}{\dfrac} command does not adapt fraction size, whereas \mintinline{tex}{\frac} does.
There is also \mintinline{tex}{\nicefrac} from \mintinline{tex}{nicefrac} package.
Outline behavior:
\begin{minted}{tex}
  \[\dfrac{1+\dfrac{4}{2}}{6}=0,5\]
  \[\frac{1+\frac{4}{2}}{6}=0,5\]
  \[\nicefrac{4}{2}=2\]
\end{minted}
\begin{mdframed}[style=example]
  \begin{align*}
    \dfrac{1+\dfrac{4}{2}}{6} & =0,5 & \frac{1+\frac{4}{2}}{6} & =0,5 & \nicefrac{4}{2} & =2
  \end{align*}
\end{mdframed}
Inline behavior:
\begin{minted}{tex}
  Some text $\dfrac{1+\dfrac{4}{2}}{6}$, more text $\frac{1+\frac{4}{2}}{6}$ and more $\nicefrac{4}{2}=2$.
\end{minted}
\begin{mdframed}[style=example]
  Some text $\dfrac{1+\dfrac{4}{2}}{6}$, more text $\frac{1+\frac{4}{2}}{6}$ and more $\nicefrac{4}{2}=2$.
\end{mdframed}

\subsection{Lengthy formulas}
\begin{minted}{tex}
  \begin{multline}
    1+2+3+4+5+6+7+\dots + \\
    + 50+51+52+53+54+55+56+57 + \dots + \\
    + 96+97+98+99+100=5050
  \end{multline}
\end{minted}
\begin{mdframed}[style=example]
  \begin{multline}
    1+2+3+4+5+6+7+\dots + \\
    + 50+51+52+53+54+55+56+57 + \dots + \\
    + 96+97+98+99+100=5050
  \end{multline}
\end{mdframed}

\subsection{Parenthesis, braces, brackets}
\begin{multicols}{2}
  \begin{minted}{tex}
    \left(\frac{x}{y}\right)
    \left[\frac{x}{y}\right]
    \left\{\frac{x}{y}\right\}
    \left<\frac{x}{y}\right>
  \end{minted}
  \newcolumn
  \begin{minted}[frame=none]{tex}
    \qty(\frac{x}{y})
    \qty[\frac{x}{y}]
    \qty{\frac{x}{y}}
    \aqty{\frac{x}{y}}
  \end{minted}
\end{multicols}
\begin{mdframed}[style=example]
  \begin{align*}
    \left(\frac{x}{y}\right)   &  &
    \left[\frac{x}{y}\right]   &  &
    \left\{\frac{x}{y}\right\} &  &
    \left<\frac{x}{y}\right>   &  &
    \qty(\frac{x}{y})          &  &
    \qty[\frac{x}{y}]          &  &
    \qty{\frac{x}{y}}          &  &
    \aqty{\frac{x}{y}}
  \end{align*}
\end{mdframed}

\begin{multicols}{2}
  \begin{minted}{tex}
    \left(\frac{x}{y}\right.
    \left[\frac{x}{y}\right.
    \left\{\frac{x}{y}\right.
    \left<\frac{x}{y}\right.
  \end{minted}
  \newcolumn
  \begin{minted}[frame=none]{tex}
    \left.\frac{x}{y}\right)
    \left.\frac{x}{y}\right]
    \left.\frac{x}{y}\right\}
    \left.\frac{x}{y}\right>
  \end{minted}
\end{multicols}
\begin{mdframed}[style=example]
  \begin{align*}
    \left(\frac{x}{y}\right.  &  &
    \left[\frac{x}{y}\right.  &  &
    \left\{\frac{x}{y}\right. &  &
    \left<\frac{x}{y}\right.  &  &
    \left.\frac{x}{y}\right)  &  &
    \left.\frac{x}{y}\right]  &  &
    \left.\frac{x}{y}\right\} &  &
    \left.\frac{x}{y}\right>
  \end{align*}
\end{mdframed}

\begin{multicols}{2}
  \begin{minted}{tex}
    \left\{
    \begin{aligned}
      1\times x & = 1 \\
      2\times y & = 4 \\
      3\times z & = 9 \\
    \end{aligned}
    \right.
  \end{minted}
  \newcolumn
  \begin{minted}[frame=none]{tex}
    |x|=
    \begin{cases}
      x,  & \text{if }  x \ge 0 \\
      -x, & \text{if } x<0
    \end{cases}
  \end{minted}
\end{multicols}

\begin{mdframed}[style=example]
  \begin{align*}
    \left\{
    \begin{aligned}
      1\times x & = 1 \\
      2\times y & = 4 \\
      3\times z & = 9 \\
    \end{aligned}
    \right.
     &  &
    |x|=
    \begin{cases}
      x,  & \text{if }  x \ge 0 \\
      -x, & \text{if } x<0
    \end{cases}
  \end{align*}
\end{mdframed}

\subsection{Matrices}
\begin{multicols}{3}
  \begin{minted}{tex}
    \begin{pmatrix}
      \xmat*{a}{3}{3}
    \end{pmatrix}
  \end{minted}
  \columnbreak
  \begin{minted}[frame=none]{tex}
    \begin{vmatrix}
      \xmat*{a}{3}{3}
    \end{vmatrix}
  \end{minted}
  \columnbreak
  \begin{minted}[frame=none]{tex}
    \begin{bmatrix}
      \xmat*{a}{3}{3}
    \end{bmatrix}
  \end{minted}
\end{multicols}

\begin{mdframed}[style=example]
  \begin{align*}
    \begin{pmatrix}
      \xmat*{a}{3}{3}
    \end{pmatrix}
     &  &
    \begin{vmatrix}
      \xmat*{a}{3}{3}
    \end{vmatrix}
     &  &
    \begin{bmatrix}
      \xmat*{a}{3}{3}
    \end{bmatrix}
  \end{align*}
\end{mdframed}

There is much more to matrices, vectors, derivatives and other common things in \mintinline{tex}{physics} package, see \href{https://ctan.org/pkg/physics}{its documentation}.

\subsection{Theorems}
For typesetting math text there are several useful environments, such as \mintinline{tex}{theorem} \autocite{Overleaf-theorems-2018}.
There is a caveat when using lists just at the beginning of these environments: lists should be escaped with \mintinline{tex}{\leavevmode} command.
\begin{minted}{tex}
  \begin{theorem}
    \leavevmode\vspace{-1.4em}
    \begin{enumerate}
      \item
      \item
    \end{enumerate}
  \end{theorem}

  \begin{proof}
    \leavevmode
    \begin{enumerate}
      \item
      \item
    \end{enumerate}
  \end{proof}
\end{minted}
\begin{mdframed}[style=example]
  \begin{theorem}
    \leavevmode\vspace{-1.4em}
    \begin{enumerate}
      \item
      \item
    \end{enumerate}
  \end{theorem}
  \begin{proof}
    \leavevmode
    \begin{enumerate}
      \item
      \item
    \end{enumerate}
  \end{proof}
\end{mdframed}

\section{Units and time}
Units are generally typesetted with \mintinline{tex}{siunitx} package:
\begin{itemize}
  \item Just unit: \mintinline{tex}{\si{\milli\meter\per\second}} --- \si{\milli\meter\per\second}.
  \item Unit with value: \mintinline{tex}{\SI{1e4}{\candela}} --- \SI{1e4}{\candela}.
  \item Values list: \mintinline{tex}{\SIlist{1;2;3}{\meter}} --- \SIlist{1;2;3}{\meter}.
  \item Numbers list: \mintinline{tex}{\numlist{1;2;3} participants} --- \numlist{1;2;3} participants.
  \item Angles: \mintinline{tex}{\ang{90}} --- \ang{90}.
\end{itemize}

Dates can be formatted with \mintinline{tex}{\DTMDate} command from \mintinline{tex}{datetime2} package, for example \mintinline{tex}{\DTMDate{2000-01-01}}:
\begin{mdframed}[style=example]
  \DTMDate{2020-01-01}
\end{mdframed}

\section{Code listings}
Code can be included in several ways:
\begin{itemize}
  \item Inline with \mintinline{tex}{\mintinline} command from \mintinline{tex}{minted} package and \mintinline{tex}{\verb} command from \mintinline{tex}{verbatim} package for some \LaTeX{} corner cases.
  \item Outline with \mintinline{tex}{minted} environment.
  \item Outline with loading code from file with \mintinline{tex}{inputminted} command.
\end{itemize}
\begin{minted}{tex}
  \inputminted{c++}{assets/cpp_example.cpp}
\end{minted}
\begin{mdframed}[style=example]
  \inputminted{c++}{assets/cpp_example.cpp}
\end{mdframed}

\section{Citations, quotations and hyperlinks}
\subsection{Hyperlinks}
Hyperlinks are typesetted with \mintinline{tex}{hyperref} package.
\begin{minted}{tex}
  Just URL: \url{https://mipt.ru/}.
  Or link: \href{https://mipt.ru/}{MIPT website}.
\end{minted}
\begin{mdframed}[style=example]
  Just URL: \url{https://mipt.ru/}.
  Or link: \href{https://mipt.ru/}{MIPT website}.
\end{mdframed}

\subsection{Quotations}
Regular quotation without reference can be done with \mintinline{tex}{\textquote} command from \mintinline{tex}{csquotes} package or standard \mintinline{tex}{quote} and \mintinline{tex}{quotation} environments:
\begin{minted}{tex}
  \textquote{This is the quoted text}
  \begin{quotation}
    This is the quoted text
  \end{quotation}
\end{minted}
\begin{mdframed}[style=example]
  \textquote{This is the quoted text}
  \begin{quotation}
    This is the quoted text
  \end{quotation}
\end{mdframed}

Quotation with an optional page number, optional automatic typesetting of dot and optional censoring:
\begin{minted}{tex}
  \textcquote[1]{Overleaf-citations-2018}[.]{Biblatex provides numerous citation styles but \textelp{unsignificant details}}
\end{minted}
\begin{mdframed}[style=example]
  \textcquote[1]{Overleaf-citations-2018}[.]{Biblatex provides numerous citation styles but \textelp{unsignificant details}}
\end{mdframed}

\subsection{Citations}
Citations (or referencing) can be done with \mintinline{tex}{\autocite} command.
It is also possible to reference with a specific part of \mintinline{tex}{biblatex} entry with \mintinline{tex}{\citeauthor}, \mintinline{tex}{\citetitle}, \mintinline{tex}{\citeurl}, \mintinline{tex}{\citeyear} or \mintinline{tex}{\citedate}.
See more at \href{https://ctan.org/pkg/biblatex}{\mintinline{tex}{biblatex} package documentation}.
Also, check out \href{https://boberle.com/projects/bibtex-entry-generator/online/}{a tool for generating \mintinline{tex}{biblatex} entries}.

\section{Counters}
Create counter:
\begin{minted}{tex}
  \newcounter{<counter_name>}[<parent_counter>]
\end{minted}
Change value:
\begin{minted}{tex}
  \setcounter{<counter_name>}{<value>}
  \addtocounter{<counter_name>}{<value>}
\end{minted}
Show counter:
\begin{minted}{tex}
  \arabic{<counter_name>}
  \roman{<counter_name>}
  \Roman{<counter_name>}
  \alph{<counter_name>}
  \Alph{<counter_name>}
  \Asbuk{<counter_name>}
  \fnsymbol{<counter_name>}
\end{minted}

\newcounter{customcounter}[section]
\setcounter{customcounter}{1}
\begin{mdframed}[style=example]
  \arabic{customcounter}
  \roman{customcounter}
  \Roman{customcounter}
  \alph{customcounter}
  \Alph{customcounter}
  \Asbuk{customcounter}
  \fnsymbol{customcounter}
\end{mdframed}

Change sections or lists numeration style:
\begin{minted}{tex}
  \renewcommand{\thesection}{\arabic{section}}
  \renewcommand{\theenumi}{\arabic{enumi}}
\end{minted}

\section{Commands, environments and logical operations}
Generic new commands and environments can be created with \mintinline[breakbefore=DC]{tex}{\NewDocumentCommand} and \mintinline[breakbefore=DE]{tex}{\NewDocumentEnvironment} commands from \mintinline{tex}{xparse} package.
Refer to \href{https://ctan.org/pkg/xparse}{its documentation} for details.

Logical operations are useful for creating different versions of the same document, for example with and without solutions.
Create bool and set it to true:
\begin{minted}{tex}
  \newbool{custombool}
  \booltrue{custombool}
  \ifbool{custombool}{True branch}{False branch}
\end{minted}
This can be combined with environment variables, for example
\begin{minted}{tex}
  \getenv[\SHLVL]{SHLVL}
  \ifdefstring{\SHLVL}{0}{SHLVL is zero}{SHLVL is not zero}
\end{minted}
\getenv[\SHLVL]{SHLVL}
\begin{mdframed}[style=example]
  \ifdefstring{\SHLVL}{0}{true}{false}
\end{mdframed}

Refer to \href{https://ctan.org/pkg/etoolbox}{\mintinline{tex}{etoolbox} package documentation} for details.

Show list of figures, tables an bibliography:
\begin{minted}{tex}
  \listoffigures
  \listoftables
  \printbibliography
\end{minted}

\listoffigures

\listoftables

\printbibliography

\end{document}
